\documentclass{article}
\usepackage{algorithm}
\usepackage{algpseudocode}
\begin{document}
\title{Pseudo code}
\maketitle
\section{Algorithm}
\begin{algorithm}
\label{alg:ogd}
\caption{Online Gradient Descent}
\begin{algorithmic}[1]
\Require parameter $\eta$, loss function $f_1,...,f_T$
\Ensure hypothesis $h_1,...,h_T$
\State Initiate $h_1=0$
\For{t=1,...,T-1}
\State $h_{t+1}=h_{t}-\eta \nabla f_t(h_t)$
\EndFor
\end{algorithmic}
\end{algorithm}


\begin{algorithm}
\label{alg:main}
\caption{}
\begin{algorithmic}[1]
\Require Budget $B$,access to online gradient descent(OGD),sequence of RSS data $d_1,,,d_T$
\Ensure the final hypothesis $\overline{h}$
\For{t=1,...,T}
\State acquire hypothesis $h_t$ from OGD
\State post price $p_t$, drawn from $P(p_t>c)=F(c)$
\If{the price get accepted}
\State send $f_t(h_t)/(1-F(c_t))$ to OGD
\Else
\State send $0$ to OGD
\EndIf 
\EndFor
\end{algorithmic}
\end{algorithm}
\section{Notes}
The form of $f_t$ is $f_t(h_t)=l(h_t,d_t)$ means that a certain kind of relation between data $d_t$ and hypothesis $h_t$, though we have not determine the specific form of $f_t$.
\end{document}